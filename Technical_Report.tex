%%%%%%%%%%%%%%%%%%%%%%%%%%%%%%%%%%%%%%%%%
% University Assignment Title Page 
% LaTeX Template
% Version 1.0 (27/12/12)
%
% This template has been downloaded from:
% http://www.LaTeXTemplates.com
%
% Original author:
% WikiBooks (http://en.wikibooks.org/wiki/LaTeX/Title_Creation)
%
% License:
% CC BY-NC-SA 3.0 (http://creativecommons.org/licenses/by-nc-sa/3.0/)
% 
% Instructions for using this template:
% This title page is capable of being compiled as is. This is not useful for 
% including it in another document. To do this, you have two options: 
%
% 1) Copy/paste everything between \begin{document} and \end{document} 
% starting at \begin{titlepage} and paste this into another LaTeX file where you 
% want your title page.
% OR
% 2) Remove everything outside the \begin{titlepage} and \end{titlepage} and 
% move this file to the same directory as the LaTeX file you wish to add it to. 
% Then add \input{./title_page_1.tex} to your LaTeX file where you want your
% title page.
%
%%%%%%%%%%%%%%%%%%%%%%%%%%%%%%%%%%%%%%%%%
%\title{Title page with logo}
%----------------------------------------------------------------------------------------
%	PACKAGES AND OTHER DOCUMENT CONFIGURATIONS
%----------------------------------------------------------------------------------------

\documentclass[12pt]{article}
\usepackage[english]{babel}
\usepackage[utf8x]{inputenc}
\usepackage{amsmath}
\usepackage{graphicx}
\usepackage[colorinlistoftodos]{todonotes}
\usepackage{upgreek}

\begin{document}

\begin{titlepage}

\newcommand{\HRule}{\rule{\linewidth}{0.5mm}} % Defines a new command for the horizontal lines, change thickness here

\center % Center everything on the page
 
%----------------------------------------------------------------------------------------
%	HEADING SECTIONS
%----------------------------------------------------------------------------------------

\textsc{\LARGE NITK SURATHKAL}\\[1.5cm] % Name of your university/college
\textsc{\Large Digital Signal Processing}\\[0.5cm] % Major heading such as course name
\textsc{\large Mini Project}\\[0.5cm] % Minor heading such as course title

%----------------------------------------------------------------------------------------
%	TITLE SECTION
%----------------------------------------------------------------------------------------

\HRule \\[0.4cm]
{ \huge \bfseries Adaptive Beamforming}\\[0.4cm] % Title of your document
\HRule \\[0.5cm]
 
%----------------------------------------------------------------------------------------
%	AUTHOR SECTION
%----------------------------------------------------------------------------------------

\begin{minipage}{0.4\textwidth}
\begin{flushleft} \large
\emph{Authors:}\\
Aniket \textsc{Rege}\\
Prabhanjan \textsc{Mannari}\\
Ritesh \textsc{Waykole}\\% Your name
\end{flushleft}
\end{minipage}
~
\begin{minipage}{0.4\textwidth}
\begin{flushright} \large
\emph{Supervisor:} \\
Dr. Pathipati \textsc{Srihari} % Supervisor's Name
\end{flushright}
\end{minipage}\\[1cm]

% If you don't want a supervisor, uncomment the two lines below and remove the section above
%\Large \emph{Author:}\\
%John \textsc{Smith}\\[3cm] % Your name

%----------------------------------------------------------------------------------------
%	DATE SECTION
%----------------------------------------------------------------------------------------

{\large \today}\\[1cm] % Date, change the \today to a set date if you want to be precise

%----------------------------------------------------------------------------------------
%	LOGO SECTION
%----------------------------------------------------------------------------------------

\includegraphics{NITK-Emblem.jpg}\\[1cm] % Include a department/university logo - this will require the graphicx package
 
%----------------------------------------------------------------------------------------

\vfill % Fill the rest of the page with whitespace

\end{titlepage}


\begin{abstract}
Adaptive array antennas use smart signal processing algorithms to allow the antenna to steer the beam to any direction of interest while simultaneously nulling interfering signals. Beamforming is the technique used to create the radiation pattern of the array by constructively adding the phases of the signals in the direction of targets and nulling the pattern of undesired/interfering targets, thus providing directional sensitivity without physically moving an array of receivers and transmitters. This can be implemented with a simple FIR tapped delay line filter, whose weights may be changed adaptively to provide optimal beamforming, which reduced the Minimum Mean Square Error (MMSE) between the actual and desired beam pattern. The most common Adaptive beamforming algorithms explored are LMS algorithm and RLS algorithm, of which the LMS algorithm is explored in this project. The LMS algorithm is described by a recursive equation which updates filter weights in such a manner that they converge to the optimum filter weight in an inverse accordance with the gradient of the mean square error vs. filter weight curve, i.e. changing the weights in a direction opposite to that of gradient slope. We hope to explore these techniques and create our own modification to existing algorithms for a more efficient beamforming process. 
\end{abstract}

\section{\LARGE Preliminaries}

The following results are given as preliminaries for the purpose of foresight into the theoretical principles behind adaptive beamforming and its use in antenna arrays.

\subsection{Random Process}

A random process X(t,s) is an ensemble of time functions from the sample space S. \\The samples $X_{ti} = x(ti), i = 1, 2, 3, \cdots n$ are n random variables characterized by their joint probability density function = $p(x_{t1}, x_{t2}, \cdots x_{tn})$

\subsection{Stationary Random Process}

Consider random process $X(ti) = x(ti), i = 1, 2, 3, \cdots n$ and \\
$X(ti + \uptau) = X(ti + \uptau, i = 1, 2, 3, \cdots n$ for some parameter $\uptau$ with probability density functions \\ 
\indent $p(x_{t1}, x_{t2}, x_{t3}, \cdots x_tn) = p_1$ \\
\indent $p(x_{t1+\uptau}, x_{t2+\uptau}, x_{t3+\uptau}, \cdots x_{tn+\uptau}) = p_2$ \\
We say the process x(ti) is stationary  if $p_1 = p_2 \forall n $

%\todo[inline, color=green!40]{This is an inline comment.}

\subsection{Statistical [Ensemble] Averages}
Consider a random process $X(t_i) = x(t_i)$ sampled at $t = t_i$\\ $X(t_i)$ is a random variable.
The $l^{th}$ moment of the random variable is defined as the expected value of $x^l(t_i)$ \\
\indent $$E[X^l(t_i)] = \int_{-\infty}^{\infty} X_{ti}^l p(x_{ti}) dx_{ti}$$
If the process is stationary, $p(x_{ti}) = p(x_{ti + \uptau})$ and hence the $l^{th}$ moment is a constant. The statistical mean or expected value of a random process is defined as $$\mu_{x}(t_i) = \int_{-\infty}^{\infty} x_{ti} p(x_{ti}) dx_{ti}$$

\subsection{Correlation}
The statistical correlation between two random variables of the process is defined as $$ E(X_{t1}X_{t2}) = \int_{-\infty}^{\infty} \int_{-\infty}^{\infty} x_{t1}x_{t2} p(x_{t1}x_{t2})dx_{t1}dx_{t2}$$

if $t_2 = t_1 + \uptau$, the above relation is known as the autocorrelation of the random process $x(t_i)$ denoted as $r_{xx}(t_1,t_{1+\uptau})$
$$r_{xx}(t_1,t_{1+\uptau}) = \int_{-\infty}^{\infty} \int_{-\infty}^{\infty} x_{t1}x_{t1 + \uptau} p(x_{t1}x_{t1 + \uptau})dx_{t1}dx_{t1 + \uptau}$$
Suppose the mean $\mu_x(t_i)$ is constant, i.e. independent of $t_i$ and the autocorrelation depends only on the lag $\uptau$ and is independent of $t_i$, then the random process $X(t_i)$ is said to be stationary in the wide sense, or \textbf {Wide Sense Stationary}.

\subsection{Auto Covariance of a Random Process}

$$ C_{xx}(t_1,t_2) = E([X_{t1}-\mu_{t1}][X_{t2}-\mu_{t2}])\\ =r_{xx}(t_1,t_2) - \mu_{t1}\mu_{t2} $$    

For a stationary random process, $e_{xx}(\uptau) = r_{xx}(\uptau) - \mu^2 $

\subsection{Statistical Averages for Joint Random Processes}
Let X(t) and Y(t) be two random processes characterized by joint PDF $P(x_{t1}, x_{t2}, \cdots y_{t1},y_{t2}, \cdots ) $ We define \\ \textbf{Cross Correlation: } $$ r_{xy}(t_1,t_2) = \int_{-\infty}^{\infty} \int_{-\infty}^{\infty} x_{t1}y_{t2} p(x_{t1}y_{t2})dx_{t1}dy_{t2}$$
\textbf{Cross Covariance: } $$ r_{xy}(t_1,t_2) = \mu_y(t_2)\mu_x(t_1)$$


$ FILL HERE MACHA



























% Commands to include a figure:
%\begin{figure}
%\centering
%\includegraphics[width=0.5\textwidth]{frog.jpg}
%\caption{\label{fig:frog}This is a figure caption.}
%\end{figure}

%\begin{table}
%\centering
%\begin{tabular}{l|r}
%Item & Quantity \\\hline
%Widgets & 42 \\
%Gadgets & 13
%\end{tabular}
%\caption{\label{tab:widgets}An example table.}
%\end{table}

\subsection{Mathematics}

\LaTeX{} is great at typesetting mathematics. Let $X_1, X_2, \ldots, X_n$ be a sequence of independent and identically distributed random variables with $\text{E}[X_i] = \mu$ and $\text{Var}[X_i] = \sigma^2 < \infty$, and let
$$S_n = \frac{X_1 + X_2 + \cdots + X_n}{n}
      = \frac{1}{n}\sum_{i}^{n} X_i$$
denote their mean. Then as $n$ approaches infinity, the random variables $\sqrt{n}(S_n - \mu)$ converge in distribution to a normal $\mathcal{N}(0, \sigma^2)$.

\subsection{Lists}

You can make lists with automatic numbering \dots

\begin{enumerate}
\item Like this,
\item and like this.
\end{enumerate}
\dots or bullet points \dots
\begin{itemize}
\item Like this,
\item and like this.
\end{itemize}

We hope you find write\LaTeX\ useful, and please let us know if you have any feedback using the help menu above.

\end{document}